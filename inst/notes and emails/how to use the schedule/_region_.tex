\message{ !name(how.to.use.schedule.tex)}\documentclass[12pt]{article}


\title{How to use the schedule file}

\begin{document}

\message{ !name(how.to.use.schedule.tex) !offset(-3) }


\maketitle

This is a quick guide to the precipitation experiment schedule
file. Files will be named FIELDSITEschedule.csv, where FIELDSITE will
be replaced with the name of your country or site
(e.g. Cardososchedule.csv, FrenchGuianaschedule.cv) The description
follows every column name:


\textbf{trt.name} = treatment name, following the naming convention
suggested by Diane.  Treatments are named after the two parameters of
the negative binomial distribution (``\emph{mu}'' and ``\emph{k}'').
Names are in the form ``muXkY'', where X and Y are numbers which
multiply \emph{mu} and \emph{k}, respectively, in different
treatments.

\textbf{intended.mu} = the actual parameter values used in this field site. the
``mu1k1'' treatment was calculated from the original data, the others
were derived from it by multiplying those values of mu and k by the
range of factors agreed upon by the working group (\emph{mu}: 0.1,
0.2, 0.4, 0.6, 0.8, 1, 1.5, 2, 2.5, 3 and \emph{k}: 0.5, 1, 2)

\textbf{intended.k} = as above, but for \emph{k}

\textbf{temporal.block} = after intended.mu and intended.k were
calculated, the treatments were divided into three random groups;
these groups begin on different days (the 1st, 2nd and 3rd days of the
experiment).  The grouping was tested with an ANOVA to confirm that
there is no significant difference in intended.mu or intended.k.

1-67 = these columns represent ``days from beginning of
experiment''. They do \textbf{not} represent calendar dates.  Each
column gives the amount of \emph{in mm} that should be added to each
plant.  Remember to multiply these by the catchment area of your
bromeliad species, AND by the correction factor for the canopy, as
described by Diane and her team in Costa Rica.  Therefore treatments
are applied to bromeliads over a period of 67 days, allowing for the
preparation of bromeliads beforehand, the calculation of correction
factors, and the final destructive sampling at the end.  plants
receive no water on days marked both NA and 0.  (NA's represent days
added by the experimental design, for example the temporal blocks, in
contrast to 0s which are predicted by our model of rainfall).

Looking along each row (rather than down each column), the pattern of treatments for each bromeliad is as follows:

0 - 2 days delay, depending on the temporal.block (marked NA)*
1 day median water of the mu1k1 treatment
3 days without water (marked NA)
60 days of treatment watering
1 day treatment median
0 - 2 days delay, depending on temporal.block (marked NA)*

*The number of temporal.block days always sums to 2 (i.e. they are either at the beginning or end of the experiment)

Thus the treatment period takes 2+1+3+60+1 = 67 days.

Plants should be sampled, as per the protocol, after the treatment median is applied.  the NA days at the end (i.e. on days 66 or 67) of the schedule are simply included to keep the schedule looking ``square''.  Do not let the bromeliads sit in the field during these days!

\end{document}

\message{ !name(how.to.use.schedule.tex) !offset(-71) }
